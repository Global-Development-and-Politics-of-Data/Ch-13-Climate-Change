% Options for packages loaded elsewhere
\PassOptionsToPackage{unicode}{hyperref}
\PassOptionsToPackage{hyphens}{url}
%
\documentclass[
  12pt,
]{article}
\usepackage{amsmath,amssymb}
\usepackage{lmodern}
\usepackage{iftex}
\ifPDFTeX
  \usepackage[T1]{fontenc}
  \usepackage[utf8]{inputenc}
  \usepackage{textcomp} % provide euro and other symbols
\else % if luatex or xetex
  \usepackage{unicode-math}
  \defaultfontfeatures{Scale=MatchLowercase}
  \defaultfontfeatures[\rmfamily]{Ligatures=TeX,Scale=1}
\fi
% Use upquote if available, for straight quotes in verbatim environments
\IfFileExists{upquote.sty}{\usepackage{upquote}}{}
\IfFileExists{microtype.sty}{% use microtype if available
  \usepackage[]{microtype}
  \UseMicrotypeSet[protrusion]{basicmath} % disable protrusion for tt fonts
}{}
\makeatletter
\@ifundefined{KOMAClassName}{% if non-KOMA class
  \IfFileExists{parskip.sty}{%
    \usepackage{parskip}
  }{% else
    \setlength{\parindent}{0pt}
    \setlength{\parskip}{6pt plus 2pt minus 1pt}}
}{% if KOMA class
  \KOMAoptions{parskip=half}}
\makeatother
\usepackage{xcolor}
\IfFileExists{xurl.sty}{\usepackage{xurl}}{} % add URL line breaks if available
\IfFileExists{bookmark.sty}{\usepackage{bookmark}}{\usepackage{hyperref}}
\hypersetup{
  hidelinks,
  pdfcreator={LaTeX via pandoc}}
\urlstyle{same} % disable monospaced font for URLs
\usepackage[margin=1in]{geometry}
\usepackage{color}
\usepackage{fancyvrb}
\newcommand{\VerbBar}{|}
\newcommand{\VERB}{\Verb[commandchars=\\\{\}]}
\DefineVerbatimEnvironment{Highlighting}{Verbatim}{commandchars=\\\{\}}
% Add ',fontsize=\small' for more characters per line
\newenvironment{Shaded}{}{}
\newcommand{\AlertTok}[1]{\textcolor[rgb]{1.00,0.00,0.00}{\textbf{#1}}}
\newcommand{\AnnotationTok}[1]{\textcolor[rgb]{0.38,0.63,0.69}{\textbf{\textit{#1}}}}
\newcommand{\AttributeTok}[1]{\textcolor[rgb]{0.49,0.56,0.16}{#1}}
\newcommand{\BaseNTok}[1]{\textcolor[rgb]{0.25,0.63,0.44}{#1}}
\newcommand{\BuiltInTok}[1]{#1}
\newcommand{\CharTok}[1]{\textcolor[rgb]{0.25,0.44,0.63}{#1}}
\newcommand{\CommentTok}[1]{\textcolor[rgb]{0.38,0.63,0.69}{\textit{#1}}}
\newcommand{\CommentVarTok}[1]{\textcolor[rgb]{0.38,0.63,0.69}{\textbf{\textit{#1}}}}
\newcommand{\ConstantTok}[1]{\textcolor[rgb]{0.53,0.00,0.00}{#1}}
\newcommand{\ControlFlowTok}[1]{\textcolor[rgb]{0.00,0.44,0.13}{\textbf{#1}}}
\newcommand{\DataTypeTok}[1]{\textcolor[rgb]{0.56,0.13,0.00}{#1}}
\newcommand{\DecValTok}[1]{\textcolor[rgb]{0.25,0.63,0.44}{#1}}
\newcommand{\DocumentationTok}[1]{\textcolor[rgb]{0.73,0.13,0.13}{\textit{#1}}}
\newcommand{\ErrorTok}[1]{\textcolor[rgb]{1.00,0.00,0.00}{\textbf{#1}}}
\newcommand{\ExtensionTok}[1]{#1}
\newcommand{\FloatTok}[1]{\textcolor[rgb]{0.25,0.63,0.44}{#1}}
\newcommand{\FunctionTok}[1]{\textcolor[rgb]{0.02,0.16,0.49}{#1}}
\newcommand{\ImportTok}[1]{#1}
\newcommand{\InformationTok}[1]{\textcolor[rgb]{0.38,0.63,0.69}{\textbf{\textit{#1}}}}
\newcommand{\KeywordTok}[1]{\textcolor[rgb]{0.00,0.44,0.13}{\textbf{#1}}}
\newcommand{\NormalTok}[1]{#1}
\newcommand{\OperatorTok}[1]{\textcolor[rgb]{0.40,0.40,0.40}{#1}}
\newcommand{\OtherTok}[1]{\textcolor[rgb]{0.00,0.44,0.13}{#1}}
\newcommand{\PreprocessorTok}[1]{\textcolor[rgb]{0.74,0.48,0.00}{#1}}
\newcommand{\RegionMarkerTok}[1]{#1}
\newcommand{\SpecialCharTok}[1]{\textcolor[rgb]{0.25,0.44,0.63}{#1}}
\newcommand{\SpecialStringTok}[1]{\textcolor[rgb]{0.73,0.40,0.53}{#1}}
\newcommand{\StringTok}[1]{\textcolor[rgb]{0.25,0.44,0.63}{#1}}
\newcommand{\VariableTok}[1]{\textcolor[rgb]{0.10,0.09,0.49}{#1}}
\newcommand{\VerbatimStringTok}[1]{\textcolor[rgb]{0.25,0.44,0.63}{#1}}
\newcommand{\WarningTok}[1]{\textcolor[rgb]{0.38,0.63,0.69}{\textbf{\textit{#1}}}}
\usepackage{graphicx}
\makeatletter
\def\maxwidth{\ifdim\Gin@nat@width>\linewidth\linewidth\else\Gin@nat@width\fi}
\def\maxheight{\ifdim\Gin@nat@height>\textheight\textheight\else\Gin@nat@height\fi}
\makeatother
% Scale images if necessary, so that they will not overflow the page
% margins by default, and it is still possible to overwrite the defaults
% using explicit options in \includegraphics[width, height, ...]{}
\setkeys{Gin}{width=\maxwidth,height=\maxheight,keepaspectratio}
% Set default figure placement to htbp
\makeatletter
\def\fps@figure{htbp}
\makeatother
\setlength{\emergencystretch}{3em} % prevent overfull lines
\providecommand{\tightlist}{%
  \setlength{\itemsep}{0pt}\setlength{\parskip}{0pt}}
\setcounter{secnumdepth}{-\maxdimen} % remove section numbering
\usepackage{amsmath}
\usepackage{mathtools}
\usepackage{amssymb}
\usepackage{makecell}
\usepackage{graphicx}
\usepackage{xcolor}
\usepackage{float}
\usepackage{fancyhdr}
\pagestyle{fancy}
\lhead{UNFCCC Emissions Data Cleaning Tutorial}
\rhead{\today}
\cfoot{\thepage}
\usepackage{caption}
\usepackage{parskip}
\setlength{\parskip}{0.75\baselineskip plus 2pt}
\captionsetup[table]{skip=10pt}
\usepackage{arydshln}
\setlength{\dashlinedash}{0.2pt}
\setlength{\dashlinegap}{4.5pt}
\usepackage{array}
\renewcommand\arraystretch{2}
\newcolumntype{P}[1]{>{\raggedright\arraybackslash}p{#1}}
\usepackage{enumitem}
\setlist[enumerate]{itemsep=0.75em,topsep=3pt}
\setlist[itemize]{itemsep=0.75em,topsep=3pt}
\usepackage{multirow}
\usepackage{setspace}
\renewcommand\maketitle{}
\usepackage{subcaption}
\ifLuaTeX
  \usepackage{selnolig}  % disable illegal ligatures
\fi
\newlength{\cslhangindent}
\setlength{\cslhangindent}{1.5em}
\newlength{\csllabelwidth}
\setlength{\csllabelwidth}{3em}
\newenvironment{CSLReferences}[2] % #1 hanging-ident, #2 entry spacing
 {% don't indent paragraphs
  \setlength{\parindent}{0pt}
  % turn on hanging indent if param 1 is 1
  \ifodd #1 \everypar{\setlength{\hangindent}{\cslhangindent}}\ignorespaces\fi
  % set entry spacing
  \ifnum #2 > 0
  \setlength{\parskip}{#2\baselineskip}
  \fi
 }%
 {}
\usepackage{calc}
\newcommand{\CSLBlock}[1]{#1\hfill\break}
\newcommand{\CSLLeftMargin}[1]{\parbox[t]{\csllabelwidth}{#1}}
\newcommand{\CSLRightInline}[1]{\parbox[t]{\linewidth - \csllabelwidth}{#1}\break}
\newcommand{\CSLIndent}[1]{\hspace{\cslhangindent}#1}

\author{}
\date{\vspace{-2.5em}}

\begin{document}

\hypertarget{install-and-load-libraries}{%
\section{Install and load libraries}\label{install-and-load-libraries}}

We need to first install all the required packages for the data cleaning
process. If you have never used the packages below, it is more likely
that you have not installed them on your machine either. Please make
sure you install each of the packages below using the following command:

\begin{Shaded}
\begin{Highlighting}[]
\FunctionTok{install.packages}\NormalTok{(}\StringTok{"here"}\NormalTok{)}
\FunctionTok{install.packages}\NormalTok{(}\StringTok{"openxlsx"}\NormalTok{)}
\FunctionTok{install.packages}\NormalTok{(}\StringTok{"janitor"}\NormalTok{)}
\FunctionTok{install.packages}\NormalTok{(}\StringTok{"tidyverse"}\NormalTok{)}
\FunctionTok{install.packages}\NormalTok{(}\StringTok{"countrycode"}\NormalTok{)}
\FunctionTok{install.packages}\NormalTok{(}\StringTok{"fuzzyjoin"}\NormalTok{)}
\FunctionTok{install.packages}\NormalTok{(}\StringTok{"knitr"}\NormalTok{)}
\end{Highlighting}
\end{Shaded}

Then you need to load the following packages:

\begin{Shaded}
\begin{Highlighting}[]
\FunctionTok{library}\NormalTok{(here)}
\FunctionTok{library}\NormalTok{(openxlsx)}
\FunctionTok{library}\NormalTok{(janitor)}
\FunctionTok{library}\NormalTok{(tidyverse)}
\FunctionTok{library}\NormalTok{(countrycode)}
\FunctionTok{library}\NormalTok{(fuzzyjoin)}
\FunctionTok{library}\NormalTok{(knitr)}
\end{Highlighting}
\end{Shaded}

\begin{itemize}
\item
  \texttt{here}: a package that reads path directories efficiently. It's
  best to use when sharing scripts or loading them in other files.
\item
  \texttt{openxlsx}: a package that reads excel files and imports them
  into R. There are other packages that do the same thing, such as
  \texttt{readxl}, but we've chosen this package because it does a
  better job at dealing with merged cells within excel sheets.
\item
  \texttt{janitor}: a package that cleans and renames variable or
  columns names into something more consistent and coherent.
\item
  \texttt{tidyverse}: a suite of R packages that deal with data
  management and general wrangling. In this document we will be using
  mostly \texttt{dplyr}, \texttt{tidyr}, and \texttt{stringr}.
\item
  \texttt{countrycode}: a package that has dataset of country names with
  their associated legal geographic information such as the U.N. System
  region names and continent names. We will use this package to merge a
  \texttt{region} column to our final clean dataset.
\item
  \texttt{fuzzyjoin}: a package to merge cells when string names are not
  fully matched, i.e.~fuzzy.
\item
  \texttt{knitr}: a package that works with R markdown and exports
  documents into different formats such as Word, PDF, or HTML. In the
  case of this document we use \texttt{knitr} to convert all the code
  chunks here into a full-length R script.
\end{itemize}

\hypertarget{data-import-and-wrangling}{%
\section{Data import and wrangling}\label{data-import-and-wrangling}}

This section is the bulk of the data cleaning process.

\hypertarget{import-data}{%
\subsection{Import data}\label{import-data}}

\hypertarget{annex-i-parties}{%
\subsubsection{Annex I parties}\label{annex-i-parties}}

\begin{Shaded}
\begin{Highlighting}[]
\NormalTok{annex\_i\_emissions }\OtherTok{\textless{}{-}}\NormalTok{ openxlsx}\SpecialCharTok{::}\FunctionTok{read.xlsx}\NormalTok{(here}\SpecialCharTok{::}\FunctionTok{here}\NormalTok{(}\StringTok{"data"}\NormalTok{, }
    \StringTok{"annual{-}net{-}emissions{-}removals{-}annex{-}i{-}raw.xlsx"}\NormalTok{), }
    \AttributeTok{startRow =} \DecValTok{5}\NormalTok{, }\AttributeTok{fillMergedCells =} \ConstantTok{TRUE}\NormalTok{, }
    \AttributeTok{rows =} \DecValTok{5}\SpecialCharTok{:}\DecValTok{96}\NormalTok{) }\SpecialCharTok{\%\textgreater{}\%}
\NormalTok{    janitor}\SpecialCharTok{::}\FunctionTok{clean\_names}\NormalTok{()}
\end{Highlighting}
\end{Shaded}

If you open the excel sheet associated with this document, you will find
that the first five rows look a bit funny and we need to find a way to
import this data without losing any of the information. We've also
noticed that there is some text at the bottom of the excel sheet which
we shouldn't import into R. This is why we specified that R only imports
rows 5 to 96.

Importing excel files into R can sometimes be more tedious because it
uses a lot of formatting functions such as merging cells in this case. R
does not know how to read merged cells and would view them as a single
cell. Therefore, we use the \texttt{openxlsx} package because it has a
function that deals with this issue.

We then use \texttt{janitor} to clean the variable names by converting
all the names to lowercase and replace empty spaces with underscores.
That way it is easier to work with these variables because R cannot read
spaces when working with objects' names.

\hypertarget{non-annex-i-parties}{%
\subsubsection{Non-Annex I parties}\label{non-annex-i-parties}}

\begin{Shaded}
\begin{Highlighting}[]
\NormalTok{non\_annex\_i\_emissions }\OtherTok{\textless{}{-}}\NormalTok{ openxlsx}\SpecialCharTok{::}\FunctionTok{read.xlsx}\NormalTok{(here}\SpecialCharTok{::}\FunctionTok{here}\NormalTok{(}\StringTok{"data"}\NormalTok{, }
    \StringTok{"annual{-}net{-}emissions{-}removals{-}non{-}annex{-}i{-}raw.xlsx"}\NormalTok{), }
    \AttributeTok{startRow =} \DecValTok{5}\NormalTok{, }\AttributeTok{fillMergedCells =} \ConstantTok{TRUE}\NormalTok{, }
    \AttributeTok{rows =} \DecValTok{5}\SpecialCharTok{:}\DecValTok{302}\NormalTok{) }\SpecialCharTok{\%\textgreater{}\%}
\NormalTok{    janitor}\SpecialCharTok{::}\FunctionTok{clean\_names}\NormalTok{()}
\end{Highlighting}
\end{Shaded}

\hypertarget{clean-variables}{%
\subsection{Clean variables}\label{clean-variables}}

Now we need to do some further cleaning and renaming of the dataset
because not everything got imported in a perfect fashion. For example,
the first row of the dataset is a whole row of only units of
measurement. We therefore do not need it and should just remove it.
Another example is that the member parties' column has an empty cell as
a header within the excel sheet. As a result, R named the empty header
\texttt{X}. As such we renamed that column to \texttt{party}. Finally,
the GHG variable did not import correctly and to save time in coding, we
decided to remove \texttt{aggregate} part of the variable name. As such
we renamed all the variables from \texttt{aggregate\_gh\_gs} to
\texttt{ghg}.

\hypertarget{annex-i-parties-1}{%
\subsubsection{Annex I parties}\label{annex-i-parties-1}}

\begin{Shaded}
\begin{Highlighting}[]
\NormalTok{annex\_i\_emissions }\OtherTok{\textless{}{-}}\NormalTok{ annex\_i\_emissions[}\SpecialCharTok{{-}}\FunctionTok{c}\NormalTok{(}\DecValTok{1}\NormalTok{), }
\NormalTok{    ]}

\FunctionTok{colnames}\NormalTok{(annex\_i\_emissions)[}\DecValTok{1}\NormalTok{] }\OtherTok{\textless{}{-}} \StringTok{"country"}

\NormalTok{annex\_i\_emissions }\OtherTok{\textless{}{-}}\NormalTok{ annex\_i\_emissions }\SpecialCharTok{\%\textgreater{}\%}
\NormalTok{    dplyr}\SpecialCharTok{::}\FunctionTok{rename\_all}\NormalTok{(}\FunctionTok{funs}\NormalTok{(stringr}\SpecialCharTok{::}\FunctionTok{str\_replace\_all}\NormalTok{(., }
        \StringTok{"aggregate\_gh\_gs"}\NormalTok{, }\StringTok{"ghg"}\NormalTok{)))}
\end{Highlighting}
\end{Shaded}

\hypertarget{non-annex-i-parties-1}{%
\subsubsection{Non-Annex I parties}\label{non-annex-i-parties-1}}

\begin{Shaded}
\begin{Highlighting}[]
\NormalTok{non\_annex\_i\_emissions }\OtherTok{\textless{}{-}}\NormalTok{ non\_annex\_i\_emissions[}\SpecialCharTok{{-}}\FunctionTok{c}\NormalTok{(}\DecValTok{1}\NormalTok{), }
\NormalTok{    ]}

\FunctionTok{colnames}\NormalTok{(non\_annex\_i\_emissions)[}\DecValTok{1}\NormalTok{] }\OtherTok{\textless{}{-}} \StringTok{"country"}

\NormalTok{non\_annex\_i\_emissions }\OtherTok{\textless{}{-}}\NormalTok{ non\_annex\_i\_emissions }\SpecialCharTok{\%\textgreater{}\%}
\NormalTok{    dplyr}\SpecialCharTok{::}\FunctionTok{rename\_all}\NormalTok{(}\FunctionTok{funs}\NormalTok{(stringr}\SpecialCharTok{::}\FunctionTok{str\_replace\_all}\NormalTok{(., }
        \StringTok{"aggregate\_gh\_gs"}\NormalTok{, }\StringTok{"ghg"}\NormalTok{)))}
\end{Highlighting}
\end{Shaded}

\hypertarget{disaggreagate-data-by-gas-type}{%
\subsection{Disaggreagate data by gas
type}\label{disaggreagate-data-by-gas-type}}

Right now the dataset is in wide format, which means that the values are
spread across countries and years instead of being associated with their
true variable which is gas type. The wide is helpful if we want to view
the dataset as whole, but can get problematic when we need it to conduct
analysis or produce visuals within R.

Another issue we had is that years were not imported into R in the
correct way. If you take look at the imported dataset, you will see that
\texttt{co2} and \texttt{aggregate\_gh\_gs} are repeated multiple times
throughout the dataset. What we do know, however is that the years begin
after column \texttt{gas} and are from ``Base year'' to 2018 for Annex I
and from 1990 to 2018 for Non-Annex I. Keeping that in mind, we first
need to split the dataset by gas type, then rename columns four to 32
(Annex I) and three to 31 (Non-Annex I) to their associated years.

\hypertarget{annex-i-parties-2}{%
\subsubsection{Annex I parties}\label{annex-i-parties-2}}

\begin{Shaded}
\begin{Highlighting}[]
\NormalTok{annex\_i\_emissions\_co2 }\OtherTok{\textless{}{-}}\NormalTok{ annex\_i\_emissions }\SpecialCharTok{\%\textgreater{}\%}
\NormalTok{    dplyr}\SpecialCharTok{::}\FunctionTok{select}\NormalTok{(}\StringTok{"country"}\NormalTok{, }\StringTok{"gas"}\NormalTok{, }\FunctionTok{contains}\NormalTok{(}\StringTok{"co2"}\NormalTok{)) }\SpecialCharTok{\%\textgreater{}\%}
\NormalTok{    dplyr}\SpecialCharTok{::}\FunctionTok{rename}\NormalTok{(}\AttributeTok{base\_year =} \StringTok{"co2"}\NormalTok{)}

\FunctionTok{colnames}\NormalTok{(annex\_i\_emissions\_co2)[}\DecValTok{4}\SpecialCharTok{:}\DecValTok{32}\NormalTok{] }\OtherTok{\textless{}{-}} \DecValTok{1990}\SpecialCharTok{:}\DecValTok{2018}

\NormalTok{annex\_i\_emissions\_ghg }\OtherTok{\textless{}{-}}\NormalTok{ annex\_i\_emissions }\SpecialCharTok{\%\textgreater{}\%}
\NormalTok{    dplyr}\SpecialCharTok{::}\FunctionTok{select}\NormalTok{(}\StringTok{"country"}\NormalTok{, }\StringTok{"gas"}\NormalTok{, }\FunctionTok{contains}\NormalTok{(}\StringTok{"ghg"}\NormalTok{)) }\SpecialCharTok{\%\textgreater{}\%}
\NormalTok{    dplyr}\SpecialCharTok{::}\FunctionTok{rename}\NormalTok{(}\AttributeTok{base\_year =} \StringTok{"ghg"}\NormalTok{)}

\FunctionTok{colnames}\NormalTok{(annex\_i\_emissions\_ghg)[}\DecValTok{4}\SpecialCharTok{:}\DecValTok{32}\NormalTok{] }\OtherTok{\textless{}{-}} \DecValTok{1990}\SpecialCharTok{:}\DecValTok{2018}
\end{Highlighting}
\end{Shaded}

\hypertarget{non-annex-i-parties-2}{%
\subsubsection{Non-Annex I parties}\label{non-annex-i-parties-2}}

\begin{Shaded}
\begin{Highlighting}[]
\NormalTok{non\_annex\_i\_emissions\_co2 }\OtherTok{\textless{}{-}}\NormalTok{ non\_annex\_i\_emissions }\SpecialCharTok{\%\textgreater{}\%}
\NormalTok{    dplyr}\SpecialCharTok{::}\FunctionTok{select}\NormalTok{(}\StringTok{"country"}\NormalTok{, }\StringTok{"gas"}\NormalTok{, }\FunctionTok{contains}\NormalTok{(}\StringTok{"co2"}\NormalTok{))}

\FunctionTok{colnames}\NormalTok{(non\_annex\_i\_emissions\_co2)[}\DecValTok{3}\SpecialCharTok{:}\DecValTok{31}\NormalTok{] }\OtherTok{\textless{}{-}} \DecValTok{1990}\SpecialCharTok{:}\DecValTok{2018}

\NormalTok{non\_annex\_i\_emissions\_ghg }\OtherTok{\textless{}{-}}\NormalTok{ non\_annex\_i\_emissions }\SpecialCharTok{\%\textgreater{}\%}
\NormalTok{    dplyr}\SpecialCharTok{::}\FunctionTok{select}\NormalTok{(}\StringTok{"country"}\NormalTok{, }\StringTok{"gas"}\NormalTok{, }\FunctionTok{contains}\NormalTok{(}\StringTok{"ghg"}\NormalTok{))}

\FunctionTok{colnames}\NormalTok{(non\_annex\_i\_emissions\_ghg)[}\DecValTok{3}\SpecialCharTok{:}\DecValTok{31}\NormalTok{] }\OtherTok{\textless{}{-}} \DecValTok{1990}\SpecialCharTok{:}\DecValTok{2018}
\end{Highlighting}
\end{Shaded}

\hypertarget{convert-datasets-to-long}{%
\subsection{Convert datasets to long}\label{convert-datasets-to-long}}

Here we convert the dataset to long. We also created a new column that
specifies each party's group name. That way it's easier to conduct
analyses by groups.

\hypertarget{annex-i-parties-3}{%
\subsubsection{Annex I parties}\label{annex-i-parties-3}}

\begin{Shaded}
\begin{Highlighting}[]
\NormalTok{annex\_i\_emissions\_co2 }\OtherTok{\textless{}{-}}\NormalTok{ annex\_i\_emissions\_co2 }\SpecialCharTok{\%\textgreater{}\%}
\NormalTok{    tidyr}\SpecialCharTok{::}\FunctionTok{gather}\NormalTok{(year, co2, base\_year}\SpecialCharTok{:}\StringTok{\textasciigrave{}}\AttributeTok{2018}\StringTok{\textasciigrave{}}\NormalTok{) }\SpecialCharTok{\%\textgreater{}\%}
    \FunctionTok{mutate}\NormalTok{(}\AttributeTok{group =} \StringTok{"Annex I"}\NormalTok{)}

\NormalTok{annex\_i\_emissions\_ghg }\OtherTok{\textless{}{-}}\NormalTok{ annex\_i\_emissions\_ghg }\SpecialCharTok{\%\textgreater{}\%}
\NormalTok{    tidyr}\SpecialCharTok{::}\FunctionTok{gather}\NormalTok{(year, ghg, base\_year}\SpecialCharTok{:}\StringTok{\textasciigrave{}}\AttributeTok{2018}\StringTok{\textasciigrave{}}\NormalTok{) }\SpecialCharTok{\%\textgreater{}\%}
    \FunctionTok{mutate}\NormalTok{(}\AttributeTok{group =} \StringTok{"Annex I"}\NormalTok{)}
\end{Highlighting}
\end{Shaded}

\hypertarget{non-annex-i-parties-3}{%
\subsubsection{Non-Annex I parties}\label{non-annex-i-parties-3}}

\begin{Shaded}
\begin{Highlighting}[]
\NormalTok{non\_annex\_i\_emissions\_co2 }\OtherTok{\textless{}{-}}\NormalTok{ non\_annex\_i\_emissions\_co2 }\SpecialCharTok{\%\textgreater{}\%}
\NormalTok{    tidyr}\SpecialCharTok{::}\FunctionTok{gather}\NormalTok{(year, co2, }\StringTok{\textasciigrave{}}\AttributeTok{1990}\StringTok{\textasciigrave{}}\SpecialCharTok{:}\StringTok{\textasciigrave{}}\AttributeTok{2018}\StringTok{\textasciigrave{}}\NormalTok{) }\SpecialCharTok{\%\textgreater{}\%}
    \FunctionTok{mutate}\NormalTok{(}\AttributeTok{group =} \StringTok{"Non{-}Annex I"}\NormalTok{)}

\NormalTok{non\_annex\_i\_emissions\_ghg }\OtherTok{\textless{}{-}}\NormalTok{ non\_annex\_i\_emissions\_ghg }\SpecialCharTok{\%\textgreater{}\%}
\NormalTok{    tidyr}\SpecialCharTok{::}\FunctionTok{gather}\NormalTok{(year, ghg, }\StringTok{\textasciigrave{}}\AttributeTok{1990}\StringTok{\textasciigrave{}}\SpecialCharTok{:}\StringTok{\textasciigrave{}}\AttributeTok{2018}\StringTok{\textasciigrave{}}\NormalTok{) }\SpecialCharTok{\%\textgreater{}\%}
    \FunctionTok{mutate}\NormalTok{(}\AttributeTok{group =} \StringTok{"Non{-}Annex I"}\NormalTok{)}
\end{Highlighting}
\end{Shaded}

\hypertarget{merge-datasets-back}{%
\subsection{Merge datasets back}\label{merge-datasets-back}}

We merge the long datasets back by columns so we can have one column for
total GHG emissions and one column for CO\textsubscript{2} only. We will
also remove the split datasets since we won't be needing them any
longer.

\hypertarget{annex-i-parties-4}{%
\subsubsection{Annex I parties}\label{annex-i-parties-4}}

\begin{Shaded}
\begin{Highlighting}[]
\NormalTok{annex\_i\_emissions }\OtherTok{\textless{}{-}}\NormalTok{ annex\_i\_emissions\_ghg }\SpecialCharTok{\%\textgreater{}\%}
\NormalTok{    dplyr}\SpecialCharTok{::}\FunctionTok{full\_join}\NormalTok{(annex\_i\_emissions\_co2, }
        \AttributeTok{by =} \FunctionTok{c}\NormalTok{(}\StringTok{"country"}\NormalTok{, }\StringTok{"gas"}\NormalTok{, }\StringTok{"year"}\NormalTok{, }
            \StringTok{"group"}\NormalTok{))}

\FunctionTok{rm}\NormalTok{(annex\_i\_emissions\_co2, annex\_i\_emissions\_ghg)}
\end{Highlighting}
\end{Shaded}

\hypertarget{non-annex-i-parties-4}{%
\subsubsection{Non-Annex I parties}\label{non-annex-i-parties-4}}

\begin{Shaded}
\begin{Highlighting}[]
\NormalTok{non\_annex\_i\_emissions }\OtherTok{\textless{}{-}}\NormalTok{ non\_annex\_i\_emissions\_ghg }\SpecialCharTok{\%\textgreater{}\%}
\NormalTok{    dplyr}\SpecialCharTok{::}\FunctionTok{full\_join}\NormalTok{(non\_annex\_i\_emissions\_co2, }
        \AttributeTok{by =} \FunctionTok{c}\NormalTok{(}\StringTok{"country"}\NormalTok{, }\StringTok{"gas"}\NormalTok{, }\StringTok{"year"}\NormalTok{, }
            \StringTok{"group"}\NormalTok{))}

\FunctionTok{rm}\NormalTok{(non\_annex\_i\_emissions\_co2, non\_annex\_i\_emissions\_ghg)}
\end{Highlighting}
\end{Shaded}

\hypertarget{merge-annex-i-and-non-annex-i}{%
\subsection{Merge Annex I and Non-Annex
I}\label{merge-annex-i-and-non-annex-i}}

Next we will merge Annex I and Non-Annex I by rows, and rename the
values for \texttt{gas} the same for both Annex I and Non-Annex I to
keep things consistent.

\begin{Shaded}
\begin{Highlighting}[]
\NormalTok{unfccc\_emissions }\OtherTok{\textless{}{-}}\NormalTok{ dplyr}\SpecialCharTok{::}\FunctionTok{bind\_rows}\NormalTok{(annex\_i\_emissions, }
\NormalTok{    non\_annex\_i\_emissions) }\SpecialCharTok{\%\textgreater{}\%}
\NormalTok{    dplyr}\SpecialCharTok{::}\FunctionTok{rename}\NormalTok{(}\AttributeTok{type =} \StringTok{"gas"}\NormalTok{) }\SpecialCharTok{\%\textgreater{}\%}
\NormalTok{    dplyr}\SpecialCharTok{::}\FunctionTok{mutate}\NormalTok{(}\AttributeTok{type =}\NormalTok{ stringr}\SpecialCharTok{::}\FunctionTok{str\_replace}\NormalTok{(type, }
        \StringTok{"Total GHG emissions excluding LULUCF/LUCF"}\NormalTok{, }
        \StringTok{"Total GHG emissions without LULUCF"}\NormalTok{)) }\SpecialCharTok{\%\textgreater{}\%}
\NormalTok{    dplyr}\SpecialCharTok{::}\FunctionTok{mutate}\NormalTok{(}\AttributeTok{type =}\NormalTok{ stringr}\SpecialCharTok{::}\FunctionTok{str\_replace}\NormalTok{(type, }
        \StringTok{"Total GHG emissions including LULUCF/LUCF"}\NormalTok{, }
        \StringTok{"Total GHG emissions with LULUCF"}\NormalTok{)) }\SpecialCharTok{\%\textgreater{}\%}
\NormalTok{    readr}\SpecialCharTok{::}\FunctionTok{type\_convert}\NormalTok{() }\SpecialCharTok{\%\textgreater{}\%}
\NormalTok{    dplyr}\SpecialCharTok{::}\FunctionTok{mutate\_if}\NormalTok{(is.character, as.factor)}
\end{Highlighting}
\end{Shaded}

\hypertarget{merge-region-data}{%
\subsection{Merge Region data}\label{merge-region-data}}

Now we will merge region dataset using the \texttt{countrycode} so we
can analyze data deprivations by region.

\begin{Shaded}
\begin{Highlighting}[]
\NormalTok{region }\OtherTok{\textless{}{-}}\NormalTok{ countrycode}\SpecialCharTok{::}\NormalTok{codelist }\SpecialCharTok{\%\textgreater{}\%}
\NormalTok{    dplyr}\SpecialCharTok{::}\FunctionTok{select}\NormalTok{(country.name.en, region, }
\NormalTok{        iso3c) }\SpecialCharTok{\%\textgreater{}\%}
\NormalTok{    dplyr}\SpecialCharTok{::}\FunctionTok{mutate}\NormalTok{(}\AttributeTok{country.name.en =}\NormalTok{ stringr}\SpecialCharTok{::}\FunctionTok{str\_replace\_all}\NormalTok{(country.name.en, }
        \FunctionTok{c}\NormalTok{(}\StringTok{\textasciigrave{}}\AttributeTok{Antigua \& Barbuda}\StringTok{\textasciigrave{}} \OtherTok{=} \StringTok{"Antigua and Barbuda"}\NormalTok{, }
            \StringTok{\textasciigrave{}}\AttributeTok{Bosnia \& Herzegovina}\StringTok{\textasciigrave{}} \OtherTok{=} \StringTok{"Bosnia and Herzegovina"}\NormalTok{, }
            \StringTok{\textasciigrave{}}\AttributeTok{Cape Verde}\StringTok{\textasciigrave{}} \OtherTok{=} \StringTok{"Cabo Verde"}\NormalTok{, }
            \StringTok{\textasciigrave{}}\AttributeTok{Congo {-} Kinshasa}\StringTok{\textasciigrave{}} \OtherTok{=} \StringTok{"Congo"}\NormalTok{, }
            \StringTok{\textasciigrave{}}\AttributeTok{Côte d’Ivoire}\StringTok{\textasciigrave{}} \OtherTok{=} \StringTok{"Cote d\textquotesingle{}Ivoire"}\NormalTok{, }
            \StringTok{\textasciigrave{}}\AttributeTok{Congo {-} Brazzaville}\StringTok{\textasciigrave{}} \OtherTok{=} \StringTok{"Democratic Republic of Congo"}\NormalTok{, }
            \AttributeTok{Laos =} \StringTok{"Lao People\textquotesingle{}s Democratic Republic"}\NormalTok{, }
            \StringTok{\textasciigrave{}}\AttributeTok{Micronesia }\SpecialCharTok{\textbackslash{}\textbackslash{}}\AttributeTok{(Federated States of}\SpecialCharTok{\textbackslash{}\textbackslash{}}\AttributeTok{)}\StringTok{\textasciigrave{}} \OtherTok{=} \StringTok{"Micronesia"}\NormalTok{, }
            \StringTok{\textasciigrave{}}\AttributeTok{Myanmar }\SpecialCharTok{\textbackslash{}\textbackslash{}}\AttributeTok{(Burma}\SpecialCharTok{\textbackslash{}\textbackslash{}}\AttributeTok{)}\StringTok{\textasciigrave{}} \OtherTok{=} \StringTok{"Myanmar"}\NormalTok{, }
            \StringTok{\textasciigrave{}}\AttributeTok{St. Lucia}\StringTok{\textasciigrave{}} \OtherTok{=} \StringTok{"Saint Lucia"}\NormalTok{, }
            \StringTok{\textasciigrave{}}\AttributeTok{St. Vincent \& Grenadines}\StringTok{\textasciigrave{}} \OtherTok{=} \StringTok{"Saint Vincent and the Grenadines"}\NormalTok{, }
            \StringTok{\textasciigrave{}}\AttributeTok{São Tomé \& Príncipe}\StringTok{\textasciigrave{}} \OtherTok{=} \StringTok{"Sao Tome and Principe"}\NormalTok{, }
            \StringTok{\textasciigrave{}}\AttributeTok{Palestinian Territories}\StringTok{\textasciigrave{}} \OtherTok{=} \StringTok{"State of Palestine"}\NormalTok{, }
            \StringTok{\textasciigrave{}}\AttributeTok{Trinidad \& Tobago}\StringTok{\textasciigrave{}} \OtherTok{=} \StringTok{"Trinidad and Tobago"}\NormalTok{, }
            \StringTok{\textasciigrave{}}\AttributeTok{North Korea}\StringTok{\textasciigrave{}} \OtherTok{=} \StringTok{"Democratic People\textquotesingle{}s Republic of Korea"}\NormalTok{, }
            \StringTok{\textasciigrave{}}\AttributeTok{South Korea}\StringTok{\textasciigrave{}} \OtherTok{=} \StringTok{"Republic of Korea"}\NormalTok{, }
            \StringTok{\textasciigrave{}}\AttributeTok{St. Kitts \& Nevis}\StringTok{\textasciigrave{}} \OtherTok{=} \StringTok{"Saint Kitts and Nevis"}\NormalTok{, }
            \AttributeTok{Vietnam =} \StringTok{"Viet Nam"}\NormalTok{)))}

\NormalTok{unfccc\_emissions }\OtherTok{\textless{}{-}}\NormalTok{ unfccc\_emissions }\SpecialCharTok{\%\textgreater{}\%}
\NormalTok{    fuzzyjoin}\SpecialCharTok{::}\FunctionTok{regex\_left\_join}\NormalTok{(region, }\AttributeTok{by =} \FunctionTok{c}\NormalTok{(}\AttributeTok{country =} \StringTok{"country.name.en"}\NormalTok{)) }\SpecialCharTok{\%\textgreater{}\%}
\NormalTok{    dplyr}\SpecialCharTok{::}\FunctionTok{rename}\NormalTok{(}\AttributeTok{iso =}\NormalTok{ iso3c) }\SpecialCharTok{\%\textgreater{}\%}
\NormalTok{    dplyr}\SpecialCharTok{::}\FunctionTok{filter}\NormalTok{(}\SpecialCharTok{!}\NormalTok{(country }\SpecialCharTok{==} \StringTok{"United Kingdom of Great Britain and Northern Ireland"} \SpecialCharTok{\&} 
\NormalTok{        iso }\SpecialCharTok{==} \StringTok{"IRL"}\NormalTok{), }\SpecialCharTok{!}\NormalTok{(country }\SpecialCharTok{==} \StringTok{"Guinea{-}Bissau"} \SpecialCharTok{\&} 
\NormalTok{        iso }\SpecialCharTok{==} \StringTok{"GIN"}\NormalTok{), }\SpecialCharTok{!}\NormalTok{(country }\SpecialCharTok{==} \StringTok{"Papua New Guinea"} \SpecialCharTok{\&} 
\NormalTok{        iso }\SpecialCharTok{==} \StringTok{"GIN"}\NormalTok{), }\SpecialCharTok{!}\NormalTok{(country }\SpecialCharTok{==} \StringTok{"Democratic People\textquotesingle{}s Republic of Korea"} \SpecialCharTok{\&} 
\NormalTok{        iso }\SpecialCharTok{==} \StringTok{"KOR"}\NormalTok{), }\SpecialCharTok{!}\NormalTok{(country }\SpecialCharTok{==} \StringTok{"Dominican Republic"} \SpecialCharTok{\&} 
\NormalTok{        iso }\SpecialCharTok{==} \StringTok{"DMA"}\NormalTok{), }\SpecialCharTok{!}\NormalTok{(country }\SpecialCharTok{==} \StringTok{"Nigeria"} \SpecialCharTok{\&} 
\NormalTok{        iso }\SpecialCharTok{==} \StringTok{"NER"}\NormalTok{), }\SpecialCharTok{!}\NormalTok{(country }\SpecialCharTok{==} \StringTok{"South Sudan"} \SpecialCharTok{\&} 
\NormalTok{        iso }\SpecialCharTok{==} \StringTok{"SDN"}\NormalTok{)) }\SpecialCharTok{\%\textgreater{}\%}
\NormalTok{    dplyr}\SpecialCharTok{::}\FunctionTok{select}\NormalTok{(}\SpecialCharTok{{-}}\NormalTok{country.name.en) }\SpecialCharTok{\%\textgreater{}\%}
\NormalTok{    dplyr}\SpecialCharTok{::}\FunctionTok{relocate}\NormalTok{(iso, }\AttributeTok{.before =}\NormalTok{ country) }\SpecialCharTok{\%\textgreater{}\%}
\NormalTok{    dplyr}\SpecialCharTok{::}\FunctionTok{relocate}\NormalTok{(}\FunctionTok{c}\NormalTok{(region, group, year), }
        \AttributeTok{.after =}\NormalTok{ country)}


\NormalTok{unfccc\_emissions}\SpecialCharTok{$}\NormalTok{region[unfccc\_emissions}\SpecialCharTok{$}\NormalTok{country }\SpecialCharTok{\%in\%} 
    \FunctionTok{c}\NormalTok{(}\StringTok{"European Union (Convention)"}\NormalTok{, }\StringTok{"European Union (KP)"}\NormalTok{)] }\OtherTok{\textless{}{-}} \StringTok{"Europe \& Central Asia"}
\end{Highlighting}
\end{Shaded}

Over here we needed to conduct some string manipulation because the
official UNFCCC dataset does not provide ISO codes, which would have
made it easier to merge the two datasets. Therefore, we needed to rename
some of the countries so we can easily join the two datasets by country
name.

\hypertarget{export-as-a-csv-file}{%
\subsection{Export as a CSV file}\label{export-as-a-csv-file}}

Now we will export our clean dataset as an CSV file so we can conduct
our analyses in the Analysis folder.

\begin{Shaded}
\begin{Highlighting}[]
\NormalTok{utils}\SpecialCharTok{::}\FunctionTok{write.csv}\NormalTok{(unfccc\_emissions, }\StringTok{"unfccc{-}emissions{-}clean.csv"}\NormalTok{, }
    \AttributeTok{row.names =} \ConstantTok{FALSE}\NormalTok{)}
\end{Highlighting}
\end{Shaded}

\hypertarget{export-as-an-r-script-for-future-use}{%
\subsection{Export as an R script for future
use}\label{export-as-an-r-script-for-future-use}}

Only run this chunk manually once within the .Rmd file. It produces an
error when knitting it as a whole because of chunk label duplicates. As
of May 10, 2021, there hasn't been a viable solution to run the code
below when as part of the knitting process.

\begin{Shaded}
\begin{Highlighting}[]
\NormalTok{knitr}\SpecialCharTok{::}\FunctionTok{purl}\NormalTok{(}\StringTok{"unfccc{-}emissions{-}clean.Rmd"}\NormalTok{, }
    \StringTok{"unfccc{-}emissions{-}clean.R"}\NormalTok{)}
\NormalTok{knitr}\SpecialCharTok{::}\FunctionTok{write\_bib}\NormalTok{(}\FunctionTok{.packages}\NormalTok{(), }\StringTok{"packages.bib"}\NormalTok{)}
\end{Highlighting}
\end{Shaded}

\hypertarget{software-used}{%
\section*{Software used}\label{software-used}}
\addcontentsline{toc}{section}{Software used}

\hypertarget{refs}{}
\begin{CSLReferences}{1}{0}
\leavevmode\vadjust pre{\hypertarget{ref-R-countrycode}{}}%
Arel-Bundock, Vincent. \emph{Countrycode: Convert Country Names and
Country Codes}, 2020.
\url{https://github.com/vincentarelbundock/countrycode}.

\leavevmode\vadjust pre{\hypertarget{ref-countrycode2018}{}}%
Arel-Bundock, Vincent, Nils Enevoldsen, and CJ Yetman. {``Countrycode:
An r Package to Convert Country Names and Country Codes.''}
\emph{Journal of Open Source Software} 3, no. 28 (2018): 848.
\url{https://doi.org/10.21105/joss.00848}.

\leavevmode\vadjust pre{\hypertarget{ref-R-janitor}{}}%
Firke, Sam. \emph{Janitor: Simple Tools for Examining and Cleaning Dirty
Data}, 2021. \url{https://github.com/sfirke/janitor}.

\leavevmode\vadjust pre{\hypertarget{ref-R-purrr}{}}%
Henry, Lionel, and Hadley Wickham. \emph{Purrr: Functional Programming
Tools}, 2020. \url{https://CRAN.R-project.org/package=purrr}.

\leavevmode\vadjust pre{\hypertarget{ref-R-here}{}}%
Müller, Kirill. \emph{Here: A Simpler Way to Find Your Files}, 2020.
\url{https://CRAN.R-project.org/package=here}.

\leavevmode\vadjust pre{\hypertarget{ref-R-tibble}{}}%
Müller, Kirill, and Hadley Wickham. \emph{Tibble: Simple Data Frames},
2021. \url{https://CRAN.R-project.org/package=tibble}.

\leavevmode\vadjust pre{\hypertarget{ref-R-base}{}}%
R Core Team. \emph{R: A Language and Environment for Statistical
Computing}. Vienna, Austria: R Foundation for Statistical Computing,
2021. \url{https://www.R-project.org/}.

\leavevmode\vadjust pre{\hypertarget{ref-R-fuzzyjoin}{}}%
Robinson, David. \emph{Fuzzyjoin: Join Tables Together on Inexact
Matching}, 2020. \url{https://github.com/dgrtwo/fuzzyjoin}.

\leavevmode\vadjust pre{\hypertarget{ref-R-openxlsx}{}}%
Schauberger, Philipp, and Alexander Walker. \emph{Openxlsx: Read, Write
and Edit Xlsx Files}, 2020.
\url{https://CRAN.R-project.org/package=openxlsx}.

\leavevmode\vadjust pre{\hypertarget{ref-R-forcats}{}}%
Wickham, Hadley. \emph{Forcats: Tools for Working with Categorical
Variables (Factors)}, 2021.
\url{https://CRAN.R-project.org/package=forcats}.

\leavevmode\vadjust pre{\hypertarget{ref-ggplot22016}{}}%
---------. \emph{Ggplot2: Elegant Graphics for Data Analysis}.
Springer-Verlag New York, 2016. \url{https://ggplot2.tidyverse.org}.

\leavevmode\vadjust pre{\hypertarget{ref-R-stringr}{}}%
---------. \emph{Stringr: Simple, Consistent Wrappers for Common String
Operations}, 2019. \url{https://CRAN.R-project.org/package=stringr}.

\leavevmode\vadjust pre{\hypertarget{ref-R-tidyr}{}}%
---------. \emph{Tidyr: Tidy Messy Data}, 2021.
\url{https://CRAN.R-project.org/package=tidyr}.

\leavevmode\vadjust pre{\hypertarget{ref-R-tidyverse}{}}%
---------. \emph{Tidyverse: Easily Install and Load the Tidyverse},
2021. \url{https://CRAN.R-project.org/package=tidyverse}.

\leavevmode\vadjust pre{\hypertarget{ref-tidyverse2019}{}}%
Wickham, Hadley, Mara Averick, Jennifer Bryan, Winston Chang, Lucy
D'Agostino McGowan, Romain François, Garrett Grolemund, et al.
{``Welcome to the {tidyverse}.''} \emph{Journal of Open Source Software}
4, no. 43 (2019): 1686. \url{https://doi.org/10.21105/joss.01686}.

\leavevmode\vadjust pre{\hypertarget{ref-R-ggplot2}{}}%
Wickham, Hadley, Winston Chang, Lionel Henry, Thomas Lin Pedersen,
Kohske Takahashi, Claus Wilke, Kara Woo, Hiroaki Yutani, and Dewey
Dunnington. \emph{Ggplot2: Create Elegant Data Visualisations Using the
Grammar of Graphics}, 2020.
\url{https://CRAN.R-project.org/package=ggplot2}.

\leavevmode\vadjust pre{\hypertarget{ref-R-dplyr}{}}%
Wickham, Hadley, Romain François, Lionel Henry, and Kirill Müller.
\emph{Dplyr: A Grammar of Data Manipulation}, 2021.
\url{https://CRAN.R-project.org/package=dplyr}.

\leavevmode\vadjust pre{\hypertarget{ref-R-readr}{}}%
Wickham, Hadley, and Jim Hester. \emph{Readr: Read Rectangular Text
Data}, 2020. \url{https://CRAN.R-project.org/package=readr}.

\leavevmode\vadjust pre{\hypertarget{ref-knitr2015}{}}%
Xie, Yihui. \emph{Dynamic Documents with {R} and Knitr}. 2nd ed. Boca
Raton, Florida: Chapman; Hall/CRC, 2015. \url{https://yihui.org/knitr/}.

\leavevmode\vadjust pre{\hypertarget{ref-knitr2014}{}}%
---------. {``Knitr: A Comprehensive Tool for Reproducible Research in
{R}.''} In \emph{Implementing Reproducible Computational Research},
edited by Victoria Stodden, Friedrich Leisch, and Roger D. Peng.
Chapman; Hall/CRC, 2014.
\url{http://www.crcpress.com/product/isbn/9781466561595}.

\leavevmode\vadjust pre{\hypertarget{ref-R-knitr}{}}%
---------. \emph{Knitr: A General-Purpose Package for Dynamic Report
Generation in r}, 2021. \url{https://yihui.org/knitr/}.

\end{CSLReferences}

\end{document}
