% Options for packages loaded elsewhere
\PassOptionsToPackage{unicode}{hyperref}
\PassOptionsToPackage{hyphens}{url}
%
\documentclass[
  12pt,
]{article}
\usepackage{amsmath,amssymb}
\usepackage{lmodern}
\usepackage{iftex}
\ifPDFTeX
  \usepackage[T1]{fontenc}
  \usepackage[utf8]{inputenc}
  \usepackage{textcomp} % provide euro and other symbols
\else % if luatex or xetex
  \usepackage{unicode-math}
  \defaultfontfeatures{Scale=MatchLowercase}
  \defaultfontfeatures[\rmfamily]{Ligatures=TeX,Scale=1}
\fi
% Use upquote if available, for straight quotes in verbatim environments
\IfFileExists{upquote.sty}{\usepackage{upquote}}{}
\IfFileExists{microtype.sty}{% use microtype if available
  \usepackage[]{microtype}
  \UseMicrotypeSet[protrusion]{basicmath} % disable protrusion for tt fonts
}{}
\makeatletter
\@ifundefined{KOMAClassName}{% if non-KOMA class
  \IfFileExists{parskip.sty}{%
    \usepackage{parskip}
  }{% else
    \setlength{\parindent}{0pt}
    \setlength{\parskip}{6pt plus 2pt minus 1pt}}
}{% if KOMA class
  \KOMAoptions{parskip=half}}
\makeatother
\usepackage{xcolor}
\IfFileExists{xurl.sty}{\usepackage{xurl}}{} % add URL line breaks if available
\IfFileExists{bookmark.sty}{\usepackage{bookmark}}{\usepackage{hyperref}}
\hypersetup{
  pdftitle={EM-DAT Data Cleaning Tutorial},
  hidelinks,
  pdfcreator={LaTeX via pandoc}}
\urlstyle{same} % disable monospaced font for URLs
\usepackage[margin=1in]{geometry}
\usepackage{color}
\usepackage{fancyvrb}
\newcommand{\VerbBar}{|}
\newcommand{\VERB}{\Verb[commandchars=\\\{\}]}
\DefineVerbatimEnvironment{Highlighting}{Verbatim}{commandchars=\\\{\}}
% Add ',fontsize=\small' for more characters per line
\newenvironment{Shaded}{}{}
\newcommand{\AlertTok}[1]{\textcolor[rgb]{1.00,0.00,0.00}{\textbf{#1}}}
\newcommand{\AnnotationTok}[1]{\textcolor[rgb]{0.38,0.63,0.69}{\textbf{\textit{#1}}}}
\newcommand{\AttributeTok}[1]{\textcolor[rgb]{0.49,0.56,0.16}{#1}}
\newcommand{\BaseNTok}[1]{\textcolor[rgb]{0.25,0.63,0.44}{#1}}
\newcommand{\BuiltInTok}[1]{#1}
\newcommand{\CharTok}[1]{\textcolor[rgb]{0.25,0.44,0.63}{#1}}
\newcommand{\CommentTok}[1]{\textcolor[rgb]{0.38,0.63,0.69}{\textit{#1}}}
\newcommand{\CommentVarTok}[1]{\textcolor[rgb]{0.38,0.63,0.69}{\textbf{\textit{#1}}}}
\newcommand{\ConstantTok}[1]{\textcolor[rgb]{0.53,0.00,0.00}{#1}}
\newcommand{\ControlFlowTok}[1]{\textcolor[rgb]{0.00,0.44,0.13}{\textbf{#1}}}
\newcommand{\DataTypeTok}[1]{\textcolor[rgb]{0.56,0.13,0.00}{#1}}
\newcommand{\DecValTok}[1]{\textcolor[rgb]{0.25,0.63,0.44}{#1}}
\newcommand{\DocumentationTok}[1]{\textcolor[rgb]{0.73,0.13,0.13}{\textit{#1}}}
\newcommand{\ErrorTok}[1]{\textcolor[rgb]{1.00,0.00,0.00}{\textbf{#1}}}
\newcommand{\ExtensionTok}[1]{#1}
\newcommand{\FloatTok}[1]{\textcolor[rgb]{0.25,0.63,0.44}{#1}}
\newcommand{\FunctionTok}[1]{\textcolor[rgb]{0.02,0.16,0.49}{#1}}
\newcommand{\ImportTok}[1]{#1}
\newcommand{\InformationTok}[1]{\textcolor[rgb]{0.38,0.63,0.69}{\textbf{\textit{#1}}}}
\newcommand{\KeywordTok}[1]{\textcolor[rgb]{0.00,0.44,0.13}{\textbf{#1}}}
\newcommand{\NormalTok}[1]{#1}
\newcommand{\OperatorTok}[1]{\textcolor[rgb]{0.40,0.40,0.40}{#1}}
\newcommand{\OtherTok}[1]{\textcolor[rgb]{0.00,0.44,0.13}{#1}}
\newcommand{\PreprocessorTok}[1]{\textcolor[rgb]{0.74,0.48,0.00}{#1}}
\newcommand{\RegionMarkerTok}[1]{#1}
\newcommand{\SpecialCharTok}[1]{\textcolor[rgb]{0.25,0.44,0.63}{#1}}
\newcommand{\SpecialStringTok}[1]{\textcolor[rgb]{0.73,0.40,0.53}{#1}}
\newcommand{\StringTok}[1]{\textcolor[rgb]{0.25,0.44,0.63}{#1}}
\newcommand{\VariableTok}[1]{\textcolor[rgb]{0.10,0.09,0.49}{#1}}
\newcommand{\VerbatimStringTok}[1]{\textcolor[rgb]{0.25,0.44,0.63}{#1}}
\newcommand{\WarningTok}[1]{\textcolor[rgb]{0.38,0.63,0.69}{\textbf{\textit{#1}}}}
\usepackage{graphicx}
\makeatletter
\def\maxwidth{\ifdim\Gin@nat@width>\linewidth\linewidth\else\Gin@nat@width\fi}
\def\maxheight{\ifdim\Gin@nat@height>\textheight\textheight\else\Gin@nat@height\fi}
\makeatother
% Scale images if necessary, so that they will not overflow the page
% margins by default, and it is still possible to overwrite the defaults
% using explicit options in \includegraphics[width, height, ...]{}
\setkeys{Gin}{width=\maxwidth,height=\maxheight,keepaspectratio}
% Set default figure placement to htbp
\makeatletter
\def\fps@figure{htbp}
\makeatother
\setlength{\emergencystretch}{3em} % prevent overfull lines
\providecommand{\tightlist}{%
  \setlength{\itemsep}{0pt}\setlength{\parskip}{0pt}}
\setcounter{secnumdepth}{-\maxdimen} % remove section numbering
\usepackage{amsmath}
\usepackage{mathtools}
\usepackage{amssymb}
\usepackage{makecell}
\usepackage{graphicx}
\usepackage{xcolor}
\usepackage{float}
\usepackage{fancyhdr}
\pagestyle{fancy}
\lhead{EM-DAT Data Cleaning Tutorial}
\rhead{\today}
\cfoot{\thepage}
\usepackage{caption}
\usepackage{parskip}
\setlength{\parskip}{0.75\baselineskip plus 2pt}
\captionsetup[table]{skip=10pt}
\usepackage{arydshln}
\setlength{\dashlinedash}{0.2pt}
\setlength{\dashlinegap}{4.5pt}
\usepackage{array}
\renewcommand\arraystretch{2}
\newcolumntype{P}[1]{>{\raggedright\arraybackslash}p{#1}}
\usepackage{enumitem}
\setlist[enumerate]{itemsep=0.75em,topsep=3pt}
\setlist[itemize]{itemsep=0.75em,topsep=3pt}
\usepackage{multirow}
\usepackage{setspace}
\renewcommand\maketitle{}
\usepackage{subcaption}
\ifLuaTeX
  \usepackage{selnolig}  % disable illegal ligatures
\fi
\newlength{\cslhangindent}
\setlength{\cslhangindent}{1.5em}
\newlength{\csllabelwidth}
\setlength{\csllabelwidth}{3em}
\newenvironment{CSLReferences}[2] % #1 hanging-ident, #2 entry spacing
 {% don't indent paragraphs
  \setlength{\parindent}{0pt}
  % turn on hanging indent if param 1 is 1
  \ifodd #1 \everypar{\setlength{\hangindent}{\cslhangindent}}\ignorespaces\fi
  % set entry spacing
  \ifnum #2 > 0
  \setlength{\parskip}{#2\baselineskip}
  \fi
 }%
 {}
\usepackage{calc}
\newcommand{\CSLBlock}[1]{#1\hfill\break}
\newcommand{\CSLLeftMargin}[1]{\parbox[t]{\csllabelwidth}{#1}}
\newcommand{\CSLRightInline}[1]{\parbox[t]{\linewidth - \csllabelwidth}{#1}\break}
\newcommand{\CSLIndent}[1]{\hspace{\cslhangindent}#1}

\title{EM-DAT Data Cleaning Tutorial}
\author{}
\date{\vspace{-2.5em}May 12, 2021}

\begin{document}
\maketitle

\hypertarget{load-libraries}{%
\subsection{Load libraries}\label{load-libraries}}

First, we need to load the following libraries:

\begin{Shaded}
\begin{Highlighting}[]
\FunctionTok{library}\NormalTok{(here)}
\FunctionTok{library}\NormalTok{(readxl)}
\FunctionTok{library}\NormalTok{(janitor)}
\FunctionTok{library}\NormalTok{(dplyr)}
\FunctionTok{library}\NormalTok{(stringr)}
\FunctionTok{library}\NormalTok{(scales)}
\end{Highlighting}
\end{Shaded}

\hypertarget{download-the-dataset}{%
\subsection{Download the dataset}\label{download-the-dataset}}

You can access the full database \href{https://public.emdat.be/}{here}
by submitting a query for the data you need. You will need to register
with the database first before you can submit a query. For the purpose
of this exercise, we will provide you with the full dataset that starts
from 1900. This dataset is as of March 24, 2021.

\hypertarget{import-dataset}{%
\subsection{Import dataset}\label{import-dataset}}

Using the package \texttt{readxl} we can import the dataset that's in
Excel format. If you preview the raw file in Excel, you will find that
the first six rows are used as description of the dataset, therefore we
need to make sure that R does not read rows that are not part of the
dataset. The function \texttt{read\_excel} provides a convenient option
called \texttt{skip}, which tells R how many rows to skip before reading
the dataset. In this case, we need to skip six rows.

\begin{Shaded}
\begin{Highlighting}[]
\NormalTok{em\_dat }\OtherTok{\textless{}{-}}\NormalTok{ readxl}\SpecialCharTok{::}\FunctionTok{read\_excel}\NormalTok{(here}\SpecialCharTok{::}\FunctionTok{here}\NormalTok{(}\StringTok{"data/em{-}dat"}\NormalTok{, }
    \StringTok{"emdat{-}public{-}2021{-}03{-}24{-}query{-}uid{-}DuX1xq{-}raw.xlsx"}\NormalTok{), }
    \AttributeTok{skip =} \DecValTok{6}\NormalTok{)}
\end{Highlighting}
\end{Shaded}

\hypertarget{clean-dataset}{%
\subsection{Clean dataset}\label{clean-dataset}}

Using the \texttt{janitor} package, we can use the function
\texttt{clean\_names} to create consistent-looking variable names.

\begin{Shaded}
\begin{Highlighting}[]
\NormalTok{em\_dat }\OtherTok{\textless{}{-}}\NormalTok{ janitor}\SpecialCharTok{::}\FunctionTok{clean\_names}\NormalTok{(em\_dat)}
\end{Highlighting}
\end{Shaded}

Instead of recreating the object \texttt{em\_dat} again, we can combine
the previous two functions using the `pipe' operator,
\texttt{\%\textgreater{}\%}, which is loaded with the \texttt{tidyverse}
package. This operator uses the previous output as the new input of the
subsequent function.

\begin{Shaded}
\begin{Highlighting}[]
\NormalTok{em\_dat }\OtherTok{\textless{}{-}}\NormalTok{ readxl}\SpecialCharTok{::}\FunctionTok{read\_excel}\NormalTok{(here}\SpecialCharTok{::}\FunctionTok{here}\NormalTok{(}\StringTok{"data/em{-}dat"}\NormalTok{, }
    \StringTok{"emdat{-}public{-}2021{-}03{-}24{-}query{-}uid{-}DuX1xq{-}raw.xlsx"}\NormalTok{), }
    \AttributeTok{skip =} \DecValTok{6}\NormalTok{) }\SpecialCharTok{\%\textgreater{}\%}
\NormalTok{    janitor}\SpecialCharTok{::}\FunctionTok{clean\_names}\NormalTok{()}
\end{Highlighting}
\end{Shaded}

In the above code chunk, we did the following:

\begin{enumerate}
\def\labelenumi{\arabic{enumi}.}
\tightlist
\item
  Imported the dataset using \texttt{read\_excel} function.
\item
  Took that dataset and cleaned all the variables' names using the
  \texttt{clean\_names} function.
\item
  Called the output of the previous two processes \texttt{em\_dat}.
\end{enumerate}

The \texttt{pipe} operator is a powerful function that can reduce the
amount of code you need to write.

\begin{Shaded}
\begin{Highlighting}[]
\NormalTok{em\_dat\_sub }\OtherTok{\textless{}{-}}\NormalTok{ em\_dat }\SpecialCharTok{\%\textgreater{}\%}
\NormalTok{    dplyr}\SpecialCharTok{::}\FunctionTok{select}\NormalTok{(iso, country, year, disaster\_type, }
\NormalTok{        total\_deaths, no\_injured, no\_affected, }
\NormalTok{        no\_homeless, total\_affected, total\_damages\_000\_us) }\SpecialCharTok{\%\textgreater{}\%}
\NormalTok{    dplyr}\SpecialCharTok{::}\FunctionTok{filter}\NormalTok{(}\FunctionTok{str\_detect}\NormalTok{(disaster\_type, }
        \StringTok{"Drought|Extreme temperature|Flood|Storm|Wildfire"}\NormalTok{))}
\end{Highlighting}
\end{Shaded}

From 1900 to 2021, 46.8\% of recorded disaster events are from the
following types: Drought, Extreme temperature, Flood, Storm, and
Wildfire. Furthermore, within the same time frame, 52.9\% of total
deaths are attributed to the disaster types that are more exacerbated by
climate change. As for people affected by these disaster, approximately
96.3\% are because of these five types of climate-related disasters.

Here we sum over disaster data by disaster type, year, and country. For
example, we count all the damages that occurred in Bangladesh for the
year 2018 by floods.

\begin{Shaded}
\begin{Highlighting}[]
\NormalTok{em\_dat\_climate }\OtherTok{\textless{}{-}}\NormalTok{ em\_dat\_sub }\SpecialCharTok{\%\textgreater{}\%}
\NormalTok{    dplyr}\SpecialCharTok{::}\FunctionTok{filter}\NormalTok{(}\FunctionTok{as.numeric}\NormalTok{(year) }\SpecialCharTok{\textgreater{}=} \DecValTok{1990} \SpecialCharTok{\&} 
        \SpecialCharTok{!}\NormalTok{(}\FunctionTok{is.na}\NormalTok{(total\_deaths) }\SpecialCharTok{\&} \FunctionTok{is.na}\NormalTok{(no\_injured) }\SpecialCharTok{\&} 
            \FunctionTok{is.na}\NormalTok{(no\_affected) }\SpecialCharTok{\&} \FunctionTok{is.na}\NormalTok{(no\_homeless) }\SpecialCharTok{\&} 
            \FunctionTok{is.na}\NormalTok{(total\_affected) }\SpecialCharTok{\&} \FunctionTok{is.na}\NormalTok{(total\_damages\_000\_us))) }\SpecialCharTok{\%\textgreater{}\%}
\NormalTok{    dplyr}\SpecialCharTok{::}\FunctionTok{group\_by}\NormalTok{(iso, country, year, disaster\_type) }\SpecialCharTok{\%\textgreater{}\%}
\NormalTok{    dplyr}\SpecialCharTok{::}\FunctionTok{summarise\_all}\NormalTok{(}\FunctionTok{funs}\NormalTok{(sum), }\AttributeTok{na.rm =} \ConstantTok{TRUE}\NormalTok{) }\SpecialCharTok{\%\textgreater{}\%}
\NormalTok{    dplyr}\SpecialCharTok{::}\FunctionTok{mutate}\NormalTok{(}\AttributeTok{country =} \FunctionTok{str\_remove}\NormalTok{(country, }
        \StringTok{" }\SpecialCharTok{\textbackslash{}\textbackslash{}}\StringTok{(the}\SpecialCharTok{\textbackslash{}\textbackslash{}}\StringTok{)"}\NormalTok{)) }\SpecialCharTok{\%\textgreater{}\%}
\NormalTok{    dplyr}\SpecialCharTok{::}\FunctionTok{mutate}\NormalTok{(}\AttributeTok{country =} \FunctionTok{str\_replace\_all}\NormalTok{(country, }
        \FunctionTok{c}\NormalTok{(}\StringTok{\textasciigrave{}}\AttributeTok{Korea }\SpecialCharTok{\textbackslash{}\textbackslash{}}\AttributeTok{(the Republic of}\SpecialCharTok{\textbackslash{}\textbackslash{}}\AttributeTok{)}\StringTok{\textasciigrave{}} \OtherTok{=} \StringTok{"Republic of Korea"}\NormalTok{, }
            \StringTok{\textasciigrave{}}\AttributeTok{Congo }\SpecialCharTok{\textbackslash{}\textbackslash{}}\AttributeTok{(the Democratic of}\SpecialCharTok{\textbackslash{}\textbackslash{}}\AttributeTok{)}\StringTok{\textasciigrave{}} \OtherTok{=} \StringTok{"Democratic Republic of the Congo"}\NormalTok{, }
            \StringTok{\textasciigrave{}}\AttributeTok{Tanzania, United Republic of}\StringTok{\textasciigrave{}} \OtherTok{=} \StringTok{"United Republic of Tanzania"}\NormalTok{, }
            \StringTok{\textasciigrave{}}\AttributeTok{Taiwan }\SpecialCharTok{\textbackslash{}\textbackslash{}}\AttributeTok{(Province of China}\SpecialCharTok{\textbackslash{}\textbackslash{}}\AttributeTok{)}\StringTok{\textasciigrave{}} \OtherTok{=} \StringTok{"China"}\NormalTok{))) }\SpecialCharTok{\%\textgreater{}\%}
\NormalTok{    dplyr}\SpecialCharTok{::}\FunctionTok{na\_if}\NormalTok{(., }\DecValTok{0}\NormalTok{)}

\NormalTok{utils}\SpecialCharTok{::}\FunctionTok{write.csv}\NormalTok{(em\_dat\_climate, }\FunctionTok{here}\NormalTok{(}\StringTok{"scripts/cleaning/em{-}dat"}\NormalTok{, }
    \StringTok{"em{-}dat{-}clean.csv"}\NormalTok{), }\AttributeTok{row.names =} \ConstantTok{FALSE}\NormalTok{)}

\FunctionTok{rm}\NormalTok{(em\_dat, em\_dat\_climate, em\_dat\_sub)}
\end{Highlighting}
\end{Shaded}

\hypertarget{export-as-an-r-script-for-future-use}{%
\subsection{Export as an R script for future
use}\label{export-as-an-r-script-for-future-use}}

Only run this chunk manually once within the .Rmd file. It produces an
error when knitting it as a whole because of chunk label duplicates. As
of May 12, 2021, there hasn't been a viable solution to run the code
below when as part of the knitting process.

\begin{Shaded}
\begin{Highlighting}[]
\NormalTok{knitr}\SpecialCharTok{::}\FunctionTok{purl}\NormalTok{(}\StringTok{"em{-}dat{-}clean.Rmd"}\NormalTok{, }\StringTok{"em{-}dat{-}clean.R"}\NormalTok{)}
\NormalTok{knitr}\SpecialCharTok{::}\FunctionTok{write\_bib}\NormalTok{(}\FunctionTok{.packages}\NormalTok{(), }\StringTok{"packages.bib"}\NormalTok{)}
\end{Highlighting}
\end{Shaded}

\hypertarget{software-used}{%
\subsection*{Software used}\label{software-used}}
\addcontentsline{toc}{subsection}{Software used}

\hypertarget{refs}{}
\begin{CSLReferences}{1}{0}
\leavevmode\vadjust pre{\hypertarget{ref-R-janitor}{}}%
Firke, Sam. \emph{Janitor: Simple Tools for Examining and Cleaning Dirty
Data}, 2021. \url{https://github.com/sfirke/janitor}.

\leavevmode\vadjust pre{\hypertarget{ref-R-here}{}}%
Müller, Kirill. \emph{Here: A Simpler Way to Find Your Files}, 2020.
\url{https://CRAN.R-project.org/package=here}.

\leavevmode\vadjust pre{\hypertarget{ref-R-base}{}}%
R Core Team. \emph{R: A Language and Environment for Statistical
Computing}. Vienna, Austria: R Foundation for Statistical Computing,
2021. \url{https://www.R-project.org/}.

\leavevmode\vadjust pre{\hypertarget{ref-R-stringr}{}}%
Wickham, Hadley. \emph{Stringr: Simple, Consistent Wrappers for Common
String Operations}, 2019.
\url{https://CRAN.R-project.org/package=stringr}.

\leavevmode\vadjust pre{\hypertarget{ref-R-readxl}{}}%
Wickham, Hadley, and Jennifer Bryan. \emph{Readxl: Read Excel Files},
2019. \url{https://CRAN.R-project.org/package=readxl}.

\leavevmode\vadjust pre{\hypertarget{ref-R-dplyr}{}}%
Wickham, Hadley, Romain François, Lionel Henry, and Kirill Müller.
\emph{Dplyr: A Grammar of Data Manipulation}, 2021.
\url{https://CRAN.R-project.org/package=dplyr}.

\leavevmode\vadjust pre{\hypertarget{ref-R-scales}{}}%
Wickham, Hadley, and Dana Seidel. \emph{Scales: Scale Functions for
Visualization}, 2020. \url{https://CRAN.R-project.org/package=scales}.

\end{CSLReferences}

\end{document}
